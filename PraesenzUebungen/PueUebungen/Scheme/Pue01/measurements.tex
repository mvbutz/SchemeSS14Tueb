\Aufgabe{Längenmaße}

In den Vereinigten Staaten wird das
  \emph{angloamerikanische} Ma\ss system (English system) verwendet, w\"ahrend in den
  meisten anderen L\"andern das \emph{metrische} System verwendet
  wird. Die folgende Tabelle gibt Umrechnungsverh\"altnisse f\"ur
  einige L\"angeneinheiten an:

 \begin{table}[h]
    \centering
    \begin{tabular}{l|l|l}
     English & & Metrisch \\
      \hline
      1 inch & & 2.54 cm \\
      1 foot & 12 in. &  \\
      1 yard & 3 ft. & \\
      1 rod & 5(1/2) yd. & \\
   \end{tabular}
    \caption{Umrechnungsfaktoren f\"ur L\"angenangaben.}
    \label{tab:umrechnung}
  \end{table}

Schreiben Sie Prozeduren für die Umrechnung von L\"angenangaben.

  \begin{enumerate}
  \item Entwickeln Sie die Umrechnungs-Prozeduren
    \texttt{inches->cm}, \texttt{feet->inches}, \texttt{yards->feet}
    sowie \texttt{rods->yards}. Binden Sie ben\"otigte Konstanten an Namen.
  \item Erstellen Sie Prozeduren, die weitere
    Einheiten umrechnen: \texttt{feet->cm},
    \texttt{yards->cm} und \texttt{rods->inches}.
    Benutzen Sie dazu nur Funktionen, die Sie
    bereits definiert haben!
  \end{enumerate}


Abgabe:
    Programm \texttt{measurements.rkt}


  \begin{solution}
\pagebreak
\lstinputlisting[caption=Längenmaße - Lösung]{../PUeUebungen/Scheme/Pue01/measurements.rkt}     
  \end{solution}


