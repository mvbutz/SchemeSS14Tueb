\Aufgabe{Bussgelder (12 Punkte)}
  
Schreiben Sie ein Programm, mit dem Sanktionen
  bei Verkehrsverst\"ossen bestimmt werden.

  \begin{enumerate}
  \item (3 Punkte) Programmieren Sie eine Prozedur
    \texttt{zu-langes-parken} f\"ur die Bewertung von zu langem Parken
    auf einem kostenpflichtigen Parkplatz. Die Prozedur bekommt eine
    Zeitspanne in Minuten \"ubergeben und gibt das entsprechende
    Verwarngeld zur\"uck.

    Diese Verwarnungen sind wie folgt festgelegt:

    \begin{itemize}
    \item \"Uberschreitung der H\"ochstparkdauer bis einschlie\ss lich 30 Minuten:
      \euro 5
    \item bis zu einer Stunde: \euro 10
    \item bis zu zwei Stunden: \euro 15
    \item bis zu drei Stunden: \euro 20
    \item l\"anger als drei Stunden: \euro 25
    \end{itemize}
    
    Test: \texttt{(check-expect (zu-langes-parken 55) 10)}

  \item (9 Punkte - jeweils 3) Das \"Uberfahren einer roten Ampel kostet je
    nach Gef\"ahrdungslage mehr, gibt Punkte und Fahrverbot. Schreiben
    Sie eine Prozedur \texttt{rote-ampel-bussgeld}, die das Bu\ss geld berechnet und eine Prozedur
    \texttt{rote-ampel-punkte} f\"ur die Punkte in
    Flensburg. Schreiben Sie außerdem eine Prozedur
    \texttt{rote-ampel-fahrverbot}, die ausgibt, ob
    ein Fahrverbot erteilt wird. \"Ubergeben Sie den Prozeduren, wie
    lange die Ampel schon rot war (in Sekunden) und ob eine
    Gef\"ahrdung oder Sachbesch\"adigung vorlag.
    
    Die Sanktionen sind wie folgt definiert:

    \begin{itemize}
    \item Bei Rot \"uber die Ampel innerhalb der ersten Sekunde:
      \euro 50 und 3 Punkte
    \item Bei Rot \"uber die Ampel innerhalb der ersten Sekunde mit
      Gef\"ahrdung oder Sachbesch\"adigung: \euro 125, 4 Punkte und
      1 Monat Fahrverbot
    \item Bei Rot \"uber die Ampel nach der ersten Sekunde: \euro
      125, 4 Punkte und 1 Monat Fahrverbot
    \item Bei Rot \"uber die Ampel nach der ersten Sekunde mit
      Gef\"ahrdung oder Sachbesch\"adigung: \euro 200, 4 Punkte und
      1 Monat Fahrverbot
    \end{itemize}
  \end{enumerate}
  
  \begin{tabular}{ll}
    Tests: & \texttt{(check-expect (rote-ampel-bussgeld 1 \#f \#t) 125)} \\
    & \texttt{(check-expect (rote-ampel-fahrverbot 15 \#f \#f) \#t)} \\
    & \texttt{(check-expect (rote-ampel-punkte 1 \#f \#f) 3)} 
  \end{tabular}\\
  
  Befolgen Sie die Konstruktionsanleitung für Prozeduren und
  Fallunterscheidungen! Achten Sie bei den Testfällen auf eine
  vollst\"andige \"Uberdeckung!\\


  \textbf{Hinweis:} Zur Lösung können Sie die eingebauten Racket-Prozeduren
    \texttt{and} und \texttt{or} verwenden.\\


Abgabe: 
    Programm \texttt{bussgeld.rkt}


  \begin{solution}
\lstinputlisting[caption=Bußgelder - Lösung]{bussgeld.rkt}     
  \end{solution}    

